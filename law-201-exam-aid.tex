\documentclass[
  letterpaper,
  landscape,
  columns=3,
  draft,
  % final,
]{cheatsheet}

\usepackage{bch-style}
\usepackage{blindtext}

\setlist{
  % Kills all vertical spacing.
  nosep,
}

\title{LAW 201 Final Exam Aid}
\author{Bryan Hoang}
\date{\today}

\begin{document}
  \maketitle{}
  \section{The Canadian Legal System}
  \subsection{Structure (3 Branches)}
  \begin{enumerate}
    \item Legislative (Pass Legislation)
    \item Executive (Implement Legislation)
    \item Judicial (Enforce Legislation)
  \end{enumerate}

  \begin{definition}[Rule of Law]
    All citizens and institutions within a country, state, or community are accountable to the same laws.
  \end{definition}

  \subsection{Case Law}
  \begin{definition}[Precedent]
    A principle or rule established in a previous legal case that is either binding on or persuasive without going to courts for a court or other tribunal when deciding subsequent cases with similar issues or facts.
  \end{definition}
  \begin{definition}[Stare decisis]
    A legal principle by which judges are obligated to respect the precedent established by prior decisions.
  \end{definition}

  \section{Public \& Constitutional Law}
  \subsection{Division of Powers}
  \begin{itemize}
    \item Validity - Does the government have the constitutional authority to enact the law?
    \item Overlap/Conflict - Can comply with both? Is federal purpose frustrated? (Operational and or Frustration conflict)
    \item Supremacy - The constitution is the supreme law.
    \item Paramountcy - Federal law is paramount when there's a conflict.
    \item Pith and substance - What's category does the law fall under? Is it enacted with the right authority then?
    \item ultra vires - An act which requires legal authority but is done without it. e.g., a province not having authority.
    \item POGG power - Federal gap filling power in drafting oversights
    \item Double aspect
  \end{itemize}
  Federal jurisdiction:
  \begin{itemize}
    \item Health
    \item Criminal Code
    \item Trade and commerce (across borders)
  \end{itemize}
  Provincial Jurisdiction:
  \begin{itemize}
    \item Health
    \item Trade and commerce (within borders)
  \end{itemize}
  \subsection{Charter of Rights and Freedoms}
  \begin{itemize}
    \item Right to free expression (sec 2b)
          \begin{itemize}
            \item Advertising (Commercial and political)
          \end{itemize}
    \item Right to equality (sec 15)
    \item Justified limits
  \end{itemize}
  \subsubsection{Oakes' Test}
  \begin{itemize}
    \item Pressing and substantial objective
    \item Rational connection that limits will advance objective. Doesn't have to be conclusive.
    \item Minimal impairment - Least intrusive way to achieve objective.
  \end{itemize}

  \section{Criminal Law}
  Types of offences:
  \begin{itemize}
    \item Summary - Least serious, within last 6 months
    \item Indictable - Most serious (e.g., aggravated assault), serious application of force
    \item Hybrid (can be either) - (e.g., assault)
    \item Regulatory - Disturbances, not true criminal offences
  \end{itemize}
  Elements of criminal offences to arrive at conviction:
  \begin{itemize}
    \item Act element in the absence of consent (actus reus)
    \item Mental element of intentionally doing so with knowledge of lack of consent (mens rea)
    \item Beyond a reasonable doubt
  \end{itemize}
  Criminal defences:
  \begin{itemize}
    \item Self-defence
    \item Duress - Compelled by threat
    \item Necessity
    \item Provocation (Partial excuse)
    \item Automontism (i.e., sleep walking)
    \item Mental disorder
  \end{itemize}

  \section{Tort Law}
  Types of torts:
  \begin{itemize}
    \item Negligence (most common)
    \item Intentional
    \item Strict Liability (trucking dynamite through city)
  \end{itemize}
  Negligence:
  \begin{itemize}
    \item Duty of care - The defendant is responsible for the care of the plaintiff.
    \item Standard of care - That of a reasonable defendant (e.g., reasonable bank)
    \item Causation (Facts)
    \item Remoteness (Legal causation) - Reasonable forsiability (e.g., plumbers and oil valve sealing leading to unfortunate delivery)
    \item Damages - Functional approach, put the person back in the position they were in before the tort, as best as money can do
  \end{itemize}
  Standard of care:
  \begin{itemize}
    \item Probability of loss
    \item Gravity of loss
    \item Burden of accident prevention
    \item General standard rather than specific defendant, who is measured under the standard, and why it was breached, regardless of the specific defendant's situation
  \end{itemize}

  \section{Contract Law}
  Formation (e.g., Carbolic Smoke Ball prize):
  \begin{itemize}
    \item Offer - Definitive terms, communicated to people (not always necessary), advertiser has intent to enter assume legal responsibility
          \begin{itemize}
            \item Is claim a mere puff?
            \item Can not accept by putting forth a counter offer.
          \end{itemize}
    \item Acceptance - Notice of acceptance not always necessary
    \item Consideration - Promise with an exchange of mutually valuable things, otherwise it's not legally binding
  \end{itemize}
  Breach:
  \begin{itemize}
    \item Enforcable agreement - the law affords a remedy for the breach
  \end{itemize}
  Remedies:
  \begin{itemize}
    \item Damages - Principle of Expectancy, that is enough \$ to put the complainer in the same position is the contract was performed (unless another opportunity comes up)
    \item Specific performance - Perform the promise
  \end{itemize}

  \section{Property Law}
  Types of property:
  \begin{itemize}
    \item Real (land)
    \item Personal (Tangible vs Intangible)
    \item Intellectual (Copyrights, trademark, patent)
    \item e.g., Can't own a spectacle (Park racing with neighbour broadcasting results)
  \end{itemize}
  Rights \& limits attached to property ownership:
  \begin{itemize}
    \item Use \& enjoyment
    \item Exploitation
    \item Alienation
    \item e.g., owning a car
  \end{itemize}
  Divison of property rights:
  \begin{itemize}
    \item Doctrine of estates
    \item Co-ownership (suvivorship)
          \begin{itemize}
            \item Join tenancy
            \item Tenancy in common
          \end{itemize}
    \item Legal \& equitable interests (trusts)
    \item Bailment (borrowing), bailor owns property, bailee is liable for damages, exercising reasonable care
    \item Lease - Right to exclusive ownership
    \item Licence - Purchase to lawfully occupy
    \item Easements - Right to use property owned by someone else for a specific purpose
    \item Covenants - Agreement between landowners
  \end{itemize}
  Posession:
  \begin{itemize}
    \item First possession (e.g., capturing/wounding)
    \item Finders rights
          \begin{itemize}
            \item Takes property into possession
            \item Finder scan't be trespassing
            \item Owners of property that demonstrate manifest intent to control access to property can assert a prior right (e.g., owning a private car vs a public parking lot)
          \end{itemize}
  \end{itemize}

  \section{Corporate Law}
  Business structures:
  \begin{itemize}
    \item Sole proprietorship
    \item Partnership
          \begin{itemize}
            \item Limited
            \item General - Every partner has all personal liability
          \end{itemize}
    \item Joint venture
    \item Trusts
    \item Corporation (need to determine a name first)
          \begin{itemize}
            \item Limits liability owners (i.e., shareholders), only corporation is responsible.
          \end{itemize}
  \end{itemize}
  6 questions:
  \begin{enumerate}
    \item Profit of not for profit?
    \item Partners?
    \item High probability of liability (risky)?
    \item Decision-making power and operational control?
    \item Main revenue src?
    \item Short-term vs long-term?
  \end{enumerate}
  Shares:
  \begin{itemize}
    \item Rights to vote, dividends, assets, and info about corporation
    \item Types of shares
          \begin{itemize}
            \item Common (voting, may have other rights at lower priority)
            \item Preferred (Dividend * asset)
          \end{itemize}
  \end{itemize}
  Corporation structure:
  \begin{itemize}
    \item Directors
          \begin{itemize}
            \item Have duties to act competently and fiduciary to the corporation (act in their best interest)
          \end{itemize}
    \item Officers
    \item Employees
  \end{itemize}
  Consumer protection:
  \begin{itemize}
    \item Competition act - Can lead to criminal sanctions
    \item False advertising
  \end{itemize}

  \section{Workplace Law}
  3 regimes:
  \begin{itemize}
    \item Common law
    \item Regulatory regime (e.g., OHSA)
    \item Collective bargaining regime
  \end{itemize}
  Employment relationships:
  \begin{itemize}
    \item Written
    \item Oral
    \item Member of a collective bargaining unit
    \item Independent contractor who is self-employed
  \end{itemize}
  \subsection{OHSA}
  \begin{itemize}
    \item Protected social areas:
          \begin{itemize}
            \item Employment
          \end{itemize}
    \item Genuine and deeply held religious beliefs in order to protect the individual
    \item Employers are required to accommodate employee's religious beliefs, but only up to the point of undue hardship.
    \item Bona fide occupational requirement(s) (BFOR)
    \item Accomodations
    \item Experience Undue Hardship (i.e., an employee quitting)
    \item Need to balance competing rights (expression religion vs right to work in an environment safe from discrimination)
    \item Poisoned environment: degrading comments made based on code grounds that influence others/their treatment. Can't be based solely on personal views, need to be objective reason for unequal terms/conditions.
    \item Can make claims against employer \& higher ups that don't try to remedy the situation before a complaint happens.
  \end{itemize}

  \section{Intellectual Property Law}
  Types of IP:
  \begin{itemize}
    \item Patent (registered, 20 yrs)
    \item Copyright (unregistered, 50 yrs after death)
    \item Trademark (optional, renewable)
  \end{itemize}
  Patents:
  \begin{itemize}
    \item The exclusivity right over an invention
    \item An invention is a new, useful, and unobvious creation, or an improvement
    \item Test for inventiveness: Would an expert in the field consider it obvious at the time of creation?
    \item Can't patent: Scientific stuff, a mere idea, surgical treatments, higher lifeforms, business concept, etc.
  \end{itemize}
  Copyright:
  \begin{itemize}
    \item Infringement based on amount and essential part of work
    \item Fair dealings exemption: research, private study, education, parody, satire, criticism, \& news reporting.
  \end{itemize}
  Trademarks:
  \begin{itemize}
    \item Common law trademark rights - Difficult to prove
    \item Can't trademark someone's name, needs to be unique. e.g, ``Jen's Soccer Academy'' is mostly descriptive/a name
    \item Passing-off - Infringement of common law trademarks
  \end{itemize}
  Trade secrets:
  \begin{itemize}
    \item e.g., Coca-Cola recipe
    \item Competitive advantage
    \item NDAs
    \item No act
  \end{itemize}

  \section{International Law}
  4 core international crimes:
  \begin{itemize}
    \item Prohibition on genocide
    \item Crimes against humanity
    \item War crimes
    \item Aggression
  \end{itemize}
  Criteria for statehood:
  \begin{itemize}
    \item Permanant population
    \item Defined territory
    \item Government
    \item Capacity to enter relations w/ other states
  \end{itemize}
  \subsection{Crime of Aggression}
  \begin{definition}[de minumus]

  \end{definition}
  \begin{itemize}
    \item Self-defence
    \item Security Council authorization
  \end{itemize}
\end{document}
