\documentclass[
  coursecode={LAW 201},
  assignmentname={ALS Task: Spot the Tort Diary},
  studentnumber=20053722,
  name={Bryan Hoang},
  draft,
  % final,
]{
  ltxanswer%
}

\usepackage[doublespacing]{setspace}

\usepackage{lipsum}
\usepackage{marginnote}
\usepackage{bch-style}

\begin{document}
  \section*{2022-02-08, Online}

  Tuesday evening, I was browsing Reddit on the subreddit r/LiveStreamFails when I came across a reference to a fairly recent event of \href{https://twitter.com/nalipls/status/1486528914902159365?s=20&t=dawuAxWiYC1T-ZoCk2uOPQ}{a tweet from a live streamer called Nalipls} dealing with a stalker for the past 2 years. The twitlonger the individual posted describes how a single person has been stalking and harassing her online since July 2020. The harassment got so serious (e.g., trying to SWAT her) that it was starting to affect her health and her job (e.g., quitting streaming). The streamer tried to get a restraining order against the stalker, but ran out of money for a lawyer before the court hearing happened.

  I believe it is a \textbf{tort of internet harassment} that was officially recognized as a new tort in the \textit{Caplan v Atas}, 2021 ONSC 670 case. The stalker is very clearly doing harm to the streamer that the streamer has tried to take legal action against.
\end{document}
